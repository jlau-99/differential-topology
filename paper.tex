\documentclass[]{article}
\usepackage[]{babel}
\usepackage{amsmath, amssymb, amsthm, graphicx}
\usepackage[shortlabels]{enumitem}

\DeclareMathOperator{\Hom}{Hom}
\DeclareMathOperator{\Det}{det}
\DeclareMathOperator{\supp}{supp}
\DeclareMathOperator{\id}{id}
\theoremstyle{definition}
\newtheorem{theorem}{Theorem}[section] % reset theorem numbering for each chapter

\theoremstyle{definition}
\newtheorem{definition}[theorem]{Definition} % definition numbers are dependent on theorem numbers

\newenvironment{sketch}{
	\renewcommand{\proofname}{Sketch of Proof}\proof}{\endproof}

\newtheorem{corollary}[theorem]{Corollary}
\newtheorem{lemma}[theorem]{Lemma}
\newtheorem{proposition}[theorem]{Proposition}
\begin{document}
\title{Differentiable Manifolds and the Hairy Ball Theorem}
\author{Jonathan Lau}
\maketitle

\section{Manifolds and Tangent Spaces}

\begin{definition}
    The upper half space is \[\mathcal{H}^n=\{(x_1, \dots, x_n)\in \mathbb{R}^n\mid x_n \geq 0\}.\] Its boundary is $\partial \mathcal{H}^n = \{(x_1, \dots, x_n)\in \mathbb{R}^n\mid x_n = 0\}$, and its interior is $\mathcal{H}^n \backslash \partial \mathcal{H}^n $.
\end{definition}

\begin{definition}[Smooth maps]
    Let $S$ be a subset of $\mathbb{R}^n$. A map $f:S \rightarrow \mathbb{R}^m$ is smooth at $p$ if there exists a neighborhood $U$ of $p$ and a smooth function $f':U \rightarrow \mathbb{R}^m$ such that $f'=f$ on $U\cap S$. If $f$ is smooth at every $p\in S$, then $f$ is smooth on $S$.
    In addition, if $F$ is bijective and $F^{-1}$ is smooth, then $F$ is called a diffeomorphism.
\end{definition}

Many properties of smooth maps on open sets also hold for smooth maps on arbitrary subsets. We will not prove them.

\begin{definition}
    Let $M$ be a Hausdorff, second countable topological space. A chart on $M$ is a pair $(U, \phi)$ where $U$ is open in $M$ and $\phi:U\rightarrow \mathcal{H}^n$ is a homeomorphism onto its image. Two charts $(U,\phi),(V,\psi)\in \mathcal{A}$ are compatible if the functions $\psi\circ\phi^{-1}$ and $\phi\circ\psi^{-1}$ are smooth on $\phi(U\cap V)$ and $\psi(U\cap V)$ respectively. An atlas $\mathcal{A}$ on $M$ is a collection of pairwise compatible charts that cover $M$.
\end{definition}


\begin{center}
    \includegraphics[scale=0.5]{compatible.png}
\end{center}

\begin{theorem}
    Each atlas is contained in a unique maximal atlas. That is, for each atlas $\mathcal{A}$ on $M$, there exists a unique atlas $\mathcal{U}$ on $M$ such that if $\mathcal{U}\subset\mathcal{U}'$, then $\mathcal{U}'=\mathcal{U}$.
\end{theorem}

\begin{proof}
    Let $\mathcal{A}$ be an atlas on $M$. Let $\mathcal{U}$ be the set of charts that are compatible with every chart in $\mathcal{A}$. Let $(U, \phi), (V, \psi)\in \mathcal{U}$, and let $p\in U\cap V$. There exists $(W, \sigma)\in \mathcal{A}$ such that $p\in W$. So, $\psi\circ\sigma^{-1}$ and $\sigma\circ\phi^{-1}$ are smooth at $\sigma(p)$ and $\phi(p)$ respectively. Therefore, \[\psi\circ\phi^{-1} = (\psi\circ\sigma^{-1})\circ(\sigma\circ\phi^{-1})\] is smooth at $\phi(p)$. As $p$ was arbitrary, $\psi\circ\phi^{-1}$ is smooth on $\phi(U\cap V)$. Similarly, $\phi\circ\psi^{-1}$ is smooth on $\psi(U\cap V)$, so $(U, \phi), (V, \psi)$ are compatible, and $\mathcal{U}$ is indeed an atlas.
    
    If $\mathcal{U}\subset \mathcal{U}'$, then every chart in $\mathcal{U}'$ is compatible with every chart in $\mathcal{U}$, in particular, with every chart in $\mathcal{A}$. By construction of $\mathcal{U}$, these charts are in $\mathcal{U}$, so $\mathcal{U}'\subset \mathcal{U}$, and $\mathcal{U}$ is maximal.

    Suppose $\mathcal{V}$ is a maximal atlas containing $\mathcal{A}$. Then, every chart in $\mathcal{V}$ is compatible with $\mathcal{A}$, so $\mathcal{V}\subset \mathcal{U}$. Similarly, $\mathcal{U}\subset \mathcal{V}$. This shows uniqueness, and concludes the proof.
\end{proof}

\begin{definition}
    A $n$ dimensional manifold $M$ is a Hausdorff, second countable topological space together with a maximal atlas.
\end{definition}

By Theorem 1.4, in order to construct a manifold, we only need to specify a topological space and an atlas. We write $M$ instead of $(M, \mathcal{A})$ for manifolds. From now on, when we say a chart on $M$, we mean a chart in $\mathcal{A}$.

\begin{theorem}[Invariance of domain]
    Let $f:U\rightarrow S$ be a diffeomorphism, where $U$ is open in $\mathbb{R}^n$ and $S\subset \mathbb{R}^n$ is an arbitrary subset. Then $S$ is open in $\mathbb{R}^n$.
\end{theorem}
\begin{proof}
    Let $p\in U$. Since $f^{-1}$ is smooth, there exists a neighborhood $V$ of $f(p)$ and a smooth function $g:V \rightarrow \mathbb{R}^n$ such that $g|_{V\cap S}=f^{-1}$.
\begin{center}
    
    \includegraphics[scale=0.5]{smooth_invariance.PNG}
\end{center}

    Then, $g\circ f$ is the identity on $f^{-1}(V)$, which is open. So, \[(Jg(f(p)))(Jf(p))=I,\] and $\Det(Jf(p))\neq 0$. By the inverse function theorem, there are neighborhoods $U_p\subset U, V_{f(p)}\subset V$ such that $f:U_p \rightarrow V_{f(p)}$ is a diffeomorphism. We also have \[V_{f(p)}=f(U_p)\subset f(U)=S.\] For each $p\in U$, we can find an open set $V_{f(p)}\mathbb{R}^n$ such that $V_{f(p)}\subset S$. So, $S$ is open.
\end{proof}

\begin{corollary}
    Let $U$ and $V$ be open subsets of $\mathcal{H}^n$, and let $f:U \rightarrow V$ be a diffeomorphism. Then, $f$ maps interior points to interior points and boundary points to boundary points.
\end{corollary}
\begin{proof} 
    Suppose $p\subset U$ is an interior point. Then, it has an neighborhood $U_p$ open in $\mathbb{R}^n$. By the above theorem, $f(p)$ lies in the open set $f(U_p)$, so $f(p)$ is an interior point. Similarly, if $f(p)$ is an interior point, $f^{-1}(f(p))=p$ is an interior point.
\end{proof}

\begin{definition}
    Let $M$ be a manifold. A point $p\in M$ is a boundary point if for some chart $(U, \phi)$, $\phi(p)\in \partial \mathcal{H}^n$. The set of all boundary points is the boundary $\partial M$.
\end{definition}

Suppose $(U,\phi)$ and $(V,\psi)$ are charts containing $p$, with $\phi(p)\in\partial \mathcal{H}^n $. By Corollary 1.7, $\psi(p)=\psi\circ\phi^{-1}(\phi(p))$ is also a boundary point. Thus, $\psi(p)$ is a boundary point for all charts $(V, \psi)$. If the boundary of a manifold is empty, then for any chart $(U, \phi)$, $\phi(U)$ is open in $\mathbb{R}^n$.

\begin{proposition}
    Let $M$ be a manifold with non empty boundary. Then, $\partial M$ is a manifold with empty boundary.
\end{proposition}
\begin{proof}
    Let $\mathcal{A}$ be an atlas on $M$. For each $(U, \phi)\in \mathcal{A}$, we construct a chart $(U\cap \partial M , \phi|_{\partial M})$ on $\partial M$. These charts are compatible, and map into $\mathbb{R}^{n-1}$, so $\partial M$ is a manifold of dimension $n-1$.
\end{proof}

\begin{definition}[Smooth Functions]
    Let $N,M$ be manifolds. A continuous function $F:N\rightarrow M$ is smooth if for all charts $(U, \phi)$ on $N$ and $(V, \psi)$ on $M$, $\psi\circ F\circ \phi^{-1}$ is smooth. If $F$ is bijective and $F^{-1}$ is also smooth, then $F$ is a diffeomorphism.
\end{definition}

\begin{center}
    \includegraphics[scale=0.5]{smooth_function.PNG}
\end{center}

Unless stated otherwise, all functions are smooth.

\begin{definition}[Partial derivatives]
    Let $f:M \rightarrow \mathbb{R}$ and $(U, x_1, \dots, x_n)$ be a chart. The partial derivative of $f$ with respect to $x_i$ at $p$ is the derivative with respect to standard coordinates
    \[\frac{\partial}{\partial x_i}\bigg|_p f=\frac{\partial}{\partial r_i}\bigg|_{\phi(p)}(f\circ\phi^{-1}).\]
\end{definition}

\begin{definition}[Germs of functions]
    Let $M$ be a manifold, and $p\in M$. Let $S$ be the set of all smooth functions defined on a neighborhood of $p$. For $f, g\in S$, we define an equivalence relation by $f\sim g$ if there is a neighborhood of $p$ on which $f=g$. These equivalence classes are the germs of $M$ at $p$, denoted $C^\infty_p(M)$.
\end{definition}

\begin{definition}
    Let $M$ be a manifold, and $p\in M$. A derivation at $p$ is a linear map $D:C^\infty_p(M) \rightarrow \mathbb{R}$ such that $D(fg)=D(f)g(p)+f(p)D(g)$.
\end{definition}

\begin{definition}
    A tangent vector at $p$ is a derivation at $p$. The tangent space of $M$ at $p$ is the set of all derivations, denoted $T_pM$.
\end{definition}

$\frac{\partial}{\partial x_1}|_p, \dots, \frac{\partial}{\partial x_n}|_p$ are examples of tangent vectors. 

\begin{definition}
    Let $F: N \rightarrow M$ be a smooth function. We define the differential of $F$ to be a function\[F_*:T_pN \rightarrow T_{F(p)}M\] defined by \[F_*(X_p)(f)=X_p(f\circ F)\quad \text{ for } f\in C^\infty_{F(p)}(M).\]
\end{definition}

It is straightforward to check that $T_pM$ is a vector space, and that $F_*$ is a linear map.

\begin{theorem}[Chain rule]
    Let $F:N \rightarrow M$, $G:M \rightarrow P$, and $p\in N$. Then, $(G\circ F)_*=G_*\circ F_*$.
\end{theorem}
\begin{proof}
    Let $X_p\in T_pN$ and $f$ be a smooth function in a neighborhood of $G(F(p))$. Then \[(G\circ F)_*(X_p)f=X_p(f\circ G\circ F)=(F_*X_p)(f\circ G)=(G_*\circ F_*(X_p))f.\]
\end{proof}

\begin{theorem}
    Let $(U, x_1, \dots, x_n)$ be a chart. Then, \[\frac{\partial}{\partial x_1}\bigg|_p, \dots, \frac{\partial}{\partial x_n}\bigg|_p\] is a basis for $T_pM$.
\end{theorem}
\begin{proof}
    By the chain rule, $\id=(\phi^{-1}\circ\phi)_*=\phi^{-1}_*\circ \phi_*$, so $\phi_*$ is an isomorphism. We see that \[\phi_*\left(\frac{\partial}{\partial x_i}\bigg|_p\right)f=\frac{\partial}{\partial x_i}\bigg|_p(f\circ\phi)=\frac{\partial}{\partial r_i}\bigg|_{\phi(p)}(f\circ\phi\circ\phi^{-1})=\frac{\partial}{\partial r_i}|_{\phi(p)}f.\] It remains to show that $\frac{\partial}{\partial r_i}|_{\phi(p)}$ is a basis of $T_{\phi(p)}\mathbb{R}^n$. For simplification, we replace $\phi(p)$ by $p$.

    Suppose $\sum_{i=1}^{n}a_i\frac{\partial}{\partial r_i}|_{p}=0$. Applying to the coordinate functions $r_i$, we see that $a_i=0$.

    Next, we show that they span the space. Let $D$ be a derivation, and $f$ a smooth function. We technically require $f$ to be a germ, but the argument still holds. By Taylor's theorem, \[f(x)=f(p)+\sum_{i=1}^{n}(x_i-p_i)g_i(x)\quad g(p)=\frac{\partial f}{\partial x_i}(p).\] Using the Leibniz rule and linearity, \[Df(x)=\sum_{i=1}^{n}Dx_ig_i(p)=\sum_{i=1}^{n}Dx_i\frac{\partial}{\partial r_i}\bigg|_{p}f.\] We have cancelled the terms $Df(p)$ and $Dp_i$ since \[D(1)=D(1\cdot 1)=1D(1)+D(1)1=2D(1),\] so $D(1)=0$ and $D(c)=0$ by linearity. Thus, \[Df=\sum_{i=1}^{n}Dx_i\frac{\partial}{\partial r_i}\bigg|_{p}.\]
\end{proof}

\begin{definition}
    Let $p\in M$. The set of all alternating multilinear functions from $(T_pM)^k$ to $\mathbb{R}$ is denoted by $\Lambda^kT_pM$. That is, for $\omega_p\in\Lambda^kT_pM$ and $v_1, \dots, v_n, w_i\in T_pM$, \begin{align*}
        \omega_p(\dots, v_i, v_{i+1}, \dots)=&-\omega_p(\dots,  v_{i+1},v_i, \dots)&1\leq i<n\\
        \omega_p(\dots, v_i+cw_i, \dots)=&\omega_p(\dots, v_i, \dots)+c\omega_p(\dots, w_i, \dots)&1\leq i\leq n
    \end{align*}
\end{definition}

\begin{definition}
    A $k$-form is a map that sends $p\in M$ to $\omega_p\in \Lambda^kT_pM$.
\end{definition}

We define $(dx_1)_p, \dots, (dx_n)_p$ to be the dual basis of $\frac{\partial}{\partial x_1}|_p, \dots, \frac{\partial}{\partial x_n}|_p$. By a result in multilinear algebra, $\{(dx_{i_1})_p\wedge\dots\wedge(dx_{i_k})_p\mid 1\leq i_1<\dots<i_n\leq n\}$ is a basis for $\Lambda^kT_pM$. If we fix a chart $(U, x_1, \dots, x_n)$, then we can write $\omega=\sum a_Idx_I$, where $a_I$ are real valued functions on $U$.

\begin{definition}
Let $\omega$ be a $k$-form. We say that $\omega$ is smooth if for every chart $(U, x_1, \dots, x_n)$, the coefficients $a_I$ in $\omega=\sum a_Idx_I$ are smooth. The set of smooth $k$-forms is denoted by $\Omega^k(M)$. The graded algebra $\bigoplus \Omega^k(M)$ is denoted $\Omega^*(M)$.
\end{definition}

\begin{definition}
    A vector field $X$ on $M$ is a function that assigns to each $p\in M$ a tangent vector $X_p\in T_pM$. A vector field is smooth (continuous) if for every chart $(U, x_1, \dots, x_n)$, the coefficients $a_i$ in $X=\sum_{i=1}^na_i\frac{\partial}{\partial x_i}|_p$ are smooth (continuous). Given a 1-form $\omega$, we define $\omega(X):M \rightarrow \mathbb{R}, p\mapsto \omega_p(X_p)$.
\end{definition}

\begin{definition}
    An exterior derivative on $M$ is an $\mathbb{R}$-linear map $\Omega^*(M) \rightarrow \Omega^*(M)$ such that \begin{enumerate}
        \item $D(\omega\cdot \tau)=(D\omega)\cdot \tau+(-1)^k\omega\cdot D\tau$ for $\omega\in \Omega^k(M)$, $\tau\in \Omega^\ell(M)$,
        \item $D\omega\in\Omega^{k+1}(M)$ for $\omega\in\Omega^k(M)$,
        \item $D\circ D=0$,
        \item if $f:M \rightarrow \mathbb{R}$ is smooth and $X$ is a smooth vector field, then $(Df)X=Xf$.
    \end{enumerate}
\end{definition}

\begin{theorem}
    Let $(U, x_1, \dots, x_n)$ be a chart containing $p$. We define an operator $d_U:\Omega^*(U) \rightarrow \Omega^*(U)$ by \[d_Uf=\sum_{i=1}^n \frac{\partial f}{\partial x_i}dx_i,\ d_U\omega=\sum_I d_Uf\wedge dx_I,\text{ where }\omega=\sum_I f_I dx_I.\] Next, we define an operator $d:\Omega^*(M)\rightarrow \Omega^*(M)$ given by $(d\omega)_p=(d_U\omega)_p$ for some chart $(U, \phi)$ containing $p$. Then, $d$ is well defined, and it is the unique exterior derivative on $M$.
\end{theorem}
\begin{proof}
    See [Tu, Theorem 19.4, p.214].
\end{proof}

\begin{definition}[Pullback]
    Let $\omega_p\in \Lambda^kT_pM$ and $F:N \rightarrow  M$. The pullback of $\omega_p$ by $F$ is $F^*(\omega_p)\in \Lambda^kT_pN$, defined as \[F^*(\omega_p)(v_1,\dots,v_k)=\omega_p(F_*(v_1),\dots, F_*(v_k)).\]
    The pullback of a differential form $\omega$ is defined as $(F^*\omega)_p=F^*(\omega_p)$.
\end{definition}

Some properties of the pullback include

\begin{proposition}
    $F^*(\omega\wedge\tau)=F^*\omega\wedge F^*\tau$.
\end{proposition}

\begin{proposition}
    $F^*d\omega=dF^*\omega$.
\end{proposition}

\begin{proposition}
    If $\omega\in \Omega^k(M)$, then $F^*\omega\in \Omega^k(N)$.
\end{proposition}

\section{Integration of Differential $n$-Forms}
From now on, we assume that for manifolds $M$ and charts $(U, \phi)$, $M$ and $U$ are path connected.

Let $\omega$, $\tau$ be non-vanishing differential $n$-forms, and $(U, \phi)$ be a chart. That is, $\omega_p$ and $\tau_p$ are not the zero maps for all $p\in M$. Then, $\omega=\sum_{i=1}^{n}f dx_1\wedge\dots\wedge dx_n$ and $\tau=\sum_{i=1}^{n}g dx_1\wedge\dots\wedge dx_n$ for smooth functions $f, g$. Since $\omega, \tau$ are non vanishing $f, g$ are non vanishing. By path connectedness and the intermediate value theorem, we see that $f$ and $g$ are either positive or negative. Furthermore, $\omega=f/g\tau$ on $U$. This holds for all charts, so $\omega=f\tau$ for some positive or negative $f$.
\begin{definition}
    Let $\omega\sim \tau$ if $\omega=f\tau$ for some positive $f$, where $\omega, \tau$ are non vanishing differential $n$-forms. This is an equivalence relation, and the equivalence classes are called orientations. If a manifold has an orientation, it is orientable. A manifold together with an orientation is an oriented manifold.
\end{definition}

\begin{definition}
    An oriented atlas is an atlas $\mathcal{A}$ such that for all $(U, \phi),(V, \psi)\in \mathcal{A}$, $\Det(J(\psi\circ\phi ^{-1}))>0$.
\end{definition}

There is actually an equivalence between orientations and equivalence classes of oriented atlases. Let $\{(U_\alpha, \phi_\alpha)\}\sim \{(V_\beta, \psi_\beta)\}$ if $\Det(J\psi_\beta\circ\phi^{-1}_\alpha)>0$ for all $\alpha, \beta$.

\begin{theorem}
    Given an orientation $[\omega]$, there exists an oriented atlas $\mathcal{A}_\omega$ such that $\omega_p(\partial_1|_p,\dots,\partial_n|_p)>0$ for all $(U, x_1, \dots, x_n)$. Given an oriented atlas, it is possible to construct a non vanishing $n$-form. Then, the identification $[\omega]\Leftrightarrow[\mathcal{A}_\omega]$ is well defined.
\end{theorem}

\begin{proof}
    See [Tu, Theorem 21.5, Theorem 21.10].
\end{proof}

\begin{definition}
    A diffeomorphism $F:N \rightarrow M$ between oriented manifolds is orientation preserving if $[F^*\omega]$ is the orientation for $N$, where $[\omega]$ is the orientation for $M$.
\end{definition}

The above definition is well defined, and is in fact equivalent to the condition that $\Det(J(\psi\circ F^\circ\phi^{-1}))>0$ for all charts $(U, \phi)$ on $N$ and $(V, \psi)$ on $M$ [Tu, Proposition 21.8, p.244].



Next, we define the orientation for the boundary.
\begin{definition}
    Let $X_p\in T_pM$, $\omega_p\in \Lambda^kT_pM$. We define a $\iota_{X_p}\omega_p\in \Lambda^{k-1}T_pM$ called the interior multiplication of $\omega_p$ with $X_p$, defined by
    \[(\iota_{X_p}\omega_p)(v_1, \dots, v_{k-1})=\omega_p(X_p, v_1, \dots, v_{k-1}).\]
\end{definition}

We also define the interior multiplication of a differential form $\omega$ with a vector field $X$ by
\[(\iota_X\omega)_p=\iota_{X_p}\omega_p.\]

\begin{proposition}
    Let $\omega_1, \dots, \omega_k\in \Lambda^1T_pM$, and $v\in T_pM$. Then \[\iota_v(\omega_1\wedge \dots\wedge \omega_k)=\sum_{i=1}^k (-1)^{k-1}\omega_i(v)\omega_1\wedge\dots\wedge\widehat{\omega_i}\wedge\dots\wedge\omega_k,\] where the $\widehat{\ \ }$ on $\omega_i$ means that $\omega_i$ is omitted from the wedge product.
\end{proposition}

\begin{proof}
    \begin{align*}
        &\iota_{v_1}(\omega_1\wedge \dots\wedge \omega_k)(v_2, \dots, v_k)\\
        =&\omega_1\wedge\dots\wedge\omega_k(v_1\dots, v_k)\\
        =&\Det(\omega_i(v_j))\\
        =&\sum_{i=1}^k(-1)^{k-1}\omega_i(v_1)\Det(\omega_\ell(v_j))_{\ell\neq i, j>1}\\
        =&\sum_{i=1}^k (-1)^{k-1}\omega_i(v_1)\omega_1\wedge\dots\wedge\widehat{\omega_i}\wedge\dots\wedge\omega_k(v_2, \dots, v_k)
    \end{align*}
\end{proof}

\begin{definition}
    A vector $v\in T_pM$ is outward pointing if for some chart $(U, \phi)$, $v=\sum a_i\partial_i|_p$, where $a_n<0$. An outward pointing vector field is a map that assigns to each $p\in \partial M$ an outward pointing vector $v_p\in T_pM$.
\end{definition}

\begin{proposition}
    Every manifold has an outward pointing vector field.
\end{proposition}
\begin{proof}
    See [Tu, Proposition 22.10, p.254].
\end{proof}

\begin{proposition}
    Let $M$ be an oriented manifold with orientation $[\omega]$, and $X$ be an outward pointing vector field. Then $\iota_X\omega$ is a non vanishing $(n-1)$-form on $\partial M$. The orientation $[\iota_X\omega]$ for $\partial M$ is well defined.
\end{proposition}

\begin{proof}
    See [Tu, Proposition 22.11, p.255].
\end{proof}

For example, $-\partial_n$ is an outward pointing vector field for $\mathcal{H}^n$. Then, the boundary of $\mathcal{H}^n$ is \[\iota_{-\partial_n}(dx_1\wedge\dots\wedge dx_n)=-\iota_{\partial_n}(dx_1\wedge\dots\wedge dx_n)=(-1)^ndx_1\wedge\dots\wedge dx_{n-1}.\]

Finally, we define integration of differential forms.

\begin{definition}
    Let $f:S \rightarrow \mathbb{R}$ for some topological space $S$. The support of $f$, $\supp(f)$, is the closure of the set $\{p\in S\mid f(p)\neq 0\}$. Let $\omega\in \Omega^k(M)$. The support of $\omega$ is the closure of the set $\{p\in M\mid \omega_p\neq 0\}$. The set of compactly supported differential $k$-forms on $M$ is denoted by $\Omega^k_c(M)$.
\end{definition}

\begin{definition}[Integration on a $\mathbb{R}^n$]
    Let $\omega\in \Omega^n_c(U)$, where $U\subset\mathbb{R}^n$. Then $\omega=f(x)dx_1\wedge\dots\wedge dx_n$ for some smooth $f$. The integral of $\omega$ over $U$ is $\int_U\omega=\int_U f$.
\end{definition}

\begin{definition}[Integration on a chart]
    Let $\omega\in \Omega^n_c(U)$ for some $(U, \phi)$. The integral of $\omega$ over $U$ is $\int_U\omega=\int_{\phi(U)}(\phi^{-1})^*\omega$. 
\end{definition}

To show that the integral does not depend on $\phi$, we first need a lemma.

\begin{lemma}
    Let $U, V$ be open in $\mathbb{R}^n$, $T:V \rightarrow U$ be a diffeomorphism, and $\omega\in \Omega^n_c(U)$. If $\Det(JT)>0$ on $V$, then $\int_{V}T^*\omega=\int_U\omega$.
\end{lemma}

\begin{proof}
    \begin{align*}
        \int_VT^*\omega=&\int_VT^*(fdx_1\wedge\dots\wedge dx_n)\\
        =&\int_V(f\circ T)T^*dx_1\wedge\dots\wedge T^*dx_n\\
        =&\int_V(f\circ T)dT^*x_1\wedge\dots\wedge dT^*x_n\\
        =&\int_V(f\circ T)dx_1\circ T\wedge\dots\wedge dx_n\circ T
    \end{align*}
    Expressing each $dx_i\circ T$ as $\sum_j \frac{\partial x_i\circ T}{\partial y_j}dy_j$, we see that \[dx_1\circ T\wedge\dots\wedge dx_n\circ T=\Det(JT)dy_1\wedge\dots\wedge dy_n.\] So, this becomes
    \begin{align*}
        &\int_V(f\circ T)\Det(JT)dy_1\wedge\dots\wedge dy_n\\
        =&\int_V(f\circ T)|\Det(JT)|dy_1\wedge\dots\wedge dy_n\\
        =&\int_V(f\circ T)|\Det(JT)|dy_1\dots dy_n\\
        =&\int_U fdx_1\dots dx_n\\
        =&\int_U\omega
    \end{align*}
\end{proof}

Let $\phi, \psi$ be two coordinate functions on $U$ such that $\Det (J(\phi\circ\psi^{-1}))>0$. By the lemma, $\int_{\phi(U)}(\phi^{-1})^*\omega=\int_{\psi(U)}(\phi\circ\psi^{-1})^*(\phi^{-1})^*\omega=\int_{\psi(U)}(\psi ^{-1})^*\omega$, so $\int_U\omega$ is well defined.

To define integration on manifolds, we need a partition of unity.

\begin{definition}
    A collection $\{A_\alpha \}$ of subsets of a topological space $S$ is said to be locally finite if every point $q$ in $S$ has a neighborhood that meets only finitely many of the sets $\{A_\alpha \}$.
\end{definition}

\begin{definition}[Partition of Unity]
    A partition of unity on a manifold $M$ is a collection of nonnegative smooth functions $\{\rho_\alpha:M \rightarrow \mathbb{R}\}_{\alpha\in A}$ such that
    \begin{enumerate}[(i)]
        \item the collection of supports, $\{\supp\rho_\alpha\}_{\alpha\in A}$, is locally finite,
        \item $\sum_{\alpha\in A} \rho_\alpha = 1.$
    \end{enumerate}
\end{definition}

\begin{definition}
    Let $\{U_\alpha\}_{\alpha\in A}$ be a collection of subsets on $M$. A partition of unity $\{\rho_\alpha\}$ is subordinate to $\{U_\alpha\}$ if $\supp\rho_\alpha\in U_\alpha$.
\end{definition}

\begin{theorem}
    Let $\{U_\alpha\}$ be an open cover of $M$. Then, there exists a partition of unity subordinate to $\{U_\alpha\}$.
\end{theorem}
\begin{proof}
    See [Tu, Theorem 13.7, p.347].
\end{proof}

\begin{proposition}
\label{finiteforms}
    Let $\{\rho_\alpha\}_{\alpha\in A}$ be a collection of functions on $M$, and $\omega$ be a differential $k$-form with compact support. If $\{\supp \rho_\alpha\}_{\alpha\in A}$ is locally finite, then $\rho_\alpha\omega\equiv0$ for all but finitely many $\alpha$.
\end{proposition}

\begin{proof}
    For each $p\in \supp\omega$, let $U_p$ be a neighborhood of $p$ that intersects finitely many $\supp\rho_\alpha$. Then the $U_p$ is an open cover of $\supp\omega$, so there is a finite subcover $U_{p_1}, \dots, U_{p_k}$. Each $U_{p_i}$ only intersects finitely many $\supp\rho_\alpha$, so $\supp\omega$ only intersects finitely many $\supp\rho_\alpha$. Finally, since \[\{p\in M\mid \rho_\alpha(p)\omega_p\}\subset\{p\in M\mid \rho_\alpha(p)\neq 0\}\cap \{p\in M\mid \omega_p\neq 0\},\] $\supp(\rho_\alpha\omega)\subset \supp\rho_\alpha\cap \supp\omega$. So, $\supp(\rho_\alpha\omega)$ is nonempty for finitely many $\alpha$.
\end{proof}

\begin{definition}[Integration on a Manifold]
    Let $\omega$ be a differential $n$-form on an oriented manifold $M$, and let $\{\rho_\alpha\}$ be a partition of unity subordinate to the oriented atlas. The integral of $\omega$ over $M$, denoted by $\int_M \omega$, is defined to be $\sum_{\alpha\in A}\int_{U_\alpha}\rho_\alpha\omega$.
\end{definition}

By Lemma \ref{finiteforms}, the sum is finite. For the definition to make sense, we need to show that the integral does not depend on the atlas and the partition of unity. Let $\{(U_\alpha, \phi_\alpha)\},\{(V_\beta, \psi_\beta)\}$ be oriented atlases and $\{\rho_\alpha\},\{\chi_\beta\}$ be subordinate partitions of unity.

\begin{align*}
    \sum_{\alpha}\int_{U_\alpha}\rho_\alpha\omega
    =&\sum_{\alpha}\int_{U_\alpha}\rho_\alpha(\sum_\beta\chi_\beta)\omega\\
    =&\sum_{\alpha}\sum_\beta\int_{U_\alpha}\rho_\alpha\chi_\beta\omega\\
    =&\sum_{\alpha}\sum_\beta\int_{U_\alpha\cap V_\beta}\rho_\alpha\chi_\beta\omega
\end{align*}
where the last equality holds since the support of $\rho_\alpha\chi_\beta$ is contained in $U_\alpha\cap V_\beta$.

To prove Stokes' theorem, we first prove it for $\mathcal{H}^n$.

\begin{theorem}
    Let $\omega$ be a smooth $(n-1)$-form on $\mathcal{H}^n$ with compact support, and $\iota:\partial M \rightarrow M$ be the inclusion map. Using $\int_{\partial \mathcal{H}^n}\omega$ as shorthand for $\int_{\partial \mathcal{H}^n}\iota^*\omega$, $\int_{\mathcal{H}^n}d\omega=\int_{\partial \mathcal{H}^n}\omega$.
\end{theorem}

\begin{proof}
    We know that $\omega=\sum f_i dx_1\wedge\dots\wedge\widehat{dx_i}\wedge\dots\wedge dx_n$. So, \begin{align*}
        d\omega=&\sum_i\left(\sum_j\frac{\partial f_i}{\partial x_j}dx_j\right)\wedge dx_1\wedge\dots\wedge\widehat{dx_i}\wedge\dots\wedge dx_n\\
        =&\sum_i\frac{\partial f_i}{\partial x_i}dx_i\wedge dx_1\wedge\dots\wedge\widehat{dx_i}\wedge\dots\wedge dx_n\\
        =&\sum_i(-1)^{i-1}\frac{\partial f_i}{\partial x_i}dx_1\wedge\dots\wedge dx_n.
    \end{align*}
    Then, \begin{align*}
        \int_{\mathcal{H}^n}d\omega=&\sum_i(-1)^{i-1}\int_{\mathcal{H}^n}\frac{\partial f_i}{\partial x_i}.
    \end{align*}
    For $i<n$, \begin{align*}
        \int_{\mathcal{H}^n}\frac{\partial f_i}{\partial x_i}=&\int_0^\infty\int_{-\infty}^\infty\dots\int_{-\infty}^\infty\frac{\partial f_i}{\partial x_i}dx_idx_1\dots dx_n+(-1)^{n-1}.
    \end{align*}
    Since $\omega$ is compactly supported, the support is contained in $[-a,a]^{n-1}\times [0,a]$ for some $a$. So, the innermost integral is
    \begin{align*}
        \int_{-\infty}^\infty\frac{\partial f_i}{\partial x_i}dx_i=&\int_{-a}^a\frac{\partial f_i}{\partial x_i}dx_i\\
        =&f_i(\dots,a,\dots)-f_i(\dots,-a, \dots)\\
        =&0
    \end{align*}
    Thus, \begin{align*}
        \int_{\mathcal{H}^n}d\omega=&(-1)^{n-1}\int_{\mathcal{H}^n}\frac{\partial f_n}{\partial x_n}\\
        =&(-1)^{n-1}\int_{\mathbb{R}^{n-1}}\int_0^a\frac{\partial f_n}{\partial x_n}dx_ndx_1\dots dx_{n-1}\\
        =&(-1)^n\int_{\mathbb{R}^{n-1}}f_n(x_1, \dots, x_{n-1}, 0)
    \end{align*}
    On the other hand, \begin{align*}
        \int_{\partial \mathcal{H}^n}\iota^*\omega=&\int_{\partial \mathcal{H}^n}\iota^*\sum f_i dx_1\wedge\dots\wedge\widehat{dx_i}\wedge\dots\wedge dx_n\\
        =&\int_{\partial \mathcal{H}^n}\sum \iota^*f_i d\iota^*x_1\wedge\dots\wedge\widehat{d\iota^*x_i}\wedge\dots\wedge d\iota^*x_n.
    \end{align*}
    But $\iota^*x_n=0$ on $\partial \mathcal{H}^n$. Writing $\iota^*x_i$ as just $x_i$, this becomes \begin{align*}\int_{\partial \mathcal{H}^n}\iota^*f_n dx_1\wedge\dots\wedge dx_{n-1}=&(-1)^n\int_{\partial \mathcal{H}^n}\iota^*f_n d(-1)^nx_1\wedge\dots\wedge dx_{n-1}\\
        =&(-1)^n\int_{\mathbb{R}^{n-1}}f_n(x_1, \dots, x_{n-1}, 0),
    \end{align*} as $(\partial \mathcal{H}^n, (-1)^nx_1, \dots, x_{n-1})$ is a chart in the oriented atlas for $\partial \mathcal{H}^n$ with the boundary orientation.
\end{proof}

Now, we prove Stokes' theorem for manifolds.

\begin{theorem}
    Let $\omega$ be a smooth $(n-1)$-form on an oriented manifold with boundary $M$. Then $\int_{M}d\omega=\int_{\partial M}\omega$.
\end{theorem}

\begin{proof}
    \begin{align*}
        \int_M d\omega=&\sum_\alpha\int_{U_\alpha}d\rho_\alpha \omega\\
        =&\sum_\alpha\int_{\phi(U_\alpha)}(\phi^{-1})^*d\rho_\alpha \omega\\
        =&\sum_\alpha\int_{\phi(U_\alpha)}d(\phi^{-1})^*\rho_\alpha \omega\\
        =&\sum_\alpha\int_{\partial\phi(U_\alpha)}\iota_{\mathcal{H}^n}^*(\phi^{-1})^*\rho_\alpha \omega\\
        =&\sum_\alpha\int_{\phi(\partial U_\alpha)}(\phi^{-1}\circ\iota_{\mathcal{H}^n})^*\rho_\alpha \omega
    \end{align*}
    Let $\phi'=(x_1|_{\partial M}, \dots, x_{n-1}|_{\partial M})$, and $\iota_M:\partial M \rightarrow M$ be the inclusion map. Since $\phi^{-1}\circ\iota_{\mathcal{H}^n}=\iota_M\circ\phi'^{-1}$, this becomes \begin{align*}
        &\sum_\alpha\int_{\phi'(\partial U_\alpha)}(\iota_M\circ\phi'^{-1})^*\rho_\alpha \omega\\
        =&\sum_\alpha\int_{\phi'(\partial U_\alpha)}(\phi'^{-1})^*\iota_M^*\rho_\alpha \omega\\
        =&\sum_\alpha\int_{\partial U_\alpha}\iota_M^*\rho_\alpha \omega\\
        =&\int_{\partial M}\omega
    \end{align*}
\end{proof}

\section{Hairy Ball Theorem}

\begin{theorem}{Hairy Ball Theorem}
    There exists a nowhere vanishing vector field on $S^n$ if and only if $n$ is odd.
\end{theorem}

To prove this, we first need a lemma on homotopies between manifolds.

\begin{definition}
    If $X$ and $Y$ are topological spaces and $F_0, F_1 : X \rightarrow  Y$ are continuous maps, a homotopy from $F_0$ to $F_1$ is a continuous map $H:X\times I \rightarrow  Y$ satisfying\begin{align*}
        H(x,0) = F_0(x)\\
        H(x,1) = F_1(x)
    \end{align*}
    for all $x\in X$, where $I$ is the unit interval $ [0,1]$. If there exists a homotopy from $F_0$ to $F_1$, we say that $F_0$ and $F_1$ are
    homotopic. If there exists a smooth homotopy $H:X\times I\rightarrow Y$ from $F_0$ to $F_1$, then we say that $F_0$ and $F_1$ are smoothly homotopic.
\end{definition}

\begin{lemma}
    \label{homotopy}
    Suppose $N$ is a manifold, $M$ is a manifold with empty boundary, and $F,G: N \rightarrow  M$ are smooth maps. If $F$ and $G$ are homotopic, then they are smoothly homotopic.
\end{lemma}

\begin{proof}
    See [Lee, Theorem 6.29, p.142].
\end{proof}

We now prove the Hairy Ball Theorem.

\begin{theorem}
    The following are equivalent:
    \begin{enumerate}
        \item There exists a nowhere-vanishing continuous vector field on $S^n$.
        \item There exists a continuous map $V : S^n \rightarrow S^n$ satisfying $V(x)\perp x$ (with respect to the Euclidean dot product on $\mathbb{R}^{n+1}$) for all $x\in S^n$.
        \item The antipodal map $\alpha:S^n \rightarrow S^n$, $x\mapsto -x$ is homotopic to $\id_{S^n}$
        \item The antipodal map $\alpha$ is orientation-preserving.
        \item $n$ is odd.
    \end{enumerate}
\end{theorem}

\begin{proof}
    $1\implies 2$: Suppose there is a nowhere vanishing vector field on $S^n$. Let $\iota:S^n \rightarrow \mathbb{R}^{n+1}$ be the inclusion map. Define $V:S^n \rightarrow \mathbb{R}^{n+1}$ by $V(p)=\iota_*(X_p)$, where we use the standard identification of $T_p\mathbb{R}^{n+1}$ with $\mathbb{R}^{n+1}$. As $X$ is non-vanishing, we can normalize $V$ to get a continuous map $V':S^n \rightarrow S^n$ satisfying $V(x)\perp x$.

    $2\implies 3$: Let $V:S^n \rightarrow S^n$ be a continuous map satisfying $V(x)\perp x$ for all $x\in S^n$. Let $H:S^n\times I\rightarrow S^n$ given by $H(x, t)=x\cos(t\pi)+V(x)\sin(t\pi)$. Then, $H(x, 0)=x=\id_{S^n}(x)$ and $H(x, 1)=-x=\alpha(x)$. Since $V$ is continuous, $H$ is continuous. Also, since $V(x)\perp x$, $|H(x, t)|^2=1$, so $H(x, t)\in S^n$. Thus, $H$ is a homotopy from $\id_{S^n}$ to $\alpha$.

    $3\implies 4$: We prove a more general version for this implication: suppose $M$ and $N$ are oriented, compact, connected, smooth manifolds,
    and $F, G:M \rightarrow N$ are homotopic diffeomorphisms. Then $F$ and $G$
    are either both orientation-preserving or both orientation-reversing. The implication $(3\Rightarrow4)$ is a direct consequence of the lemma.
    
    To prove this, first note that by Lemma \ref{homotopy}, there is a smooth homotopy $H:M\times I \rightarrow N$ from $F$ to $G$. Let $[\omega]$ be the orientation for $N$. Then, since $d\omega=0$, so \begin{align*}
        0=&\int_{M\times I}H^*d\omega\\
        =&\int_{M\times I}dH^*\omega\\
        =&\int_{\partial M\times I}H^*\omega\\
        =&\int_{M\times \{0\}}H^*\omega+\int_{M\times \{1\}}H^*\omega\\
        =&\pm\left(\int_M G^*\omega-\int_M F^*\omega\right)
    \end{align*}
    where the $\pm$ comes from which orientation we choose for $M\times I$. The application of Stokes' theorem requires that the forms have compact support, which they do as they are non-vanishing on compact manifolds. So, $\int_M G^*\omega=\int_M F^*\omega$. They cannot be equal to 0 as $M$ is connected. So, they are either both positive or both negative, which means that $[F^*\omega]=[G^*\omega]$ and concludes the proof for the lemma.

    $4\implies 5$: Let $B^{n+1}$ be the closed unit ball in $\mathbb{R}^{n+1}$, $A:B^{n+1} \rightarrow B^{n+1}$ be the map $x \mapsto -x$, $\Omega=dx_1\wedge\dots\wedge dx_{n+1}$, and $X=\sum x_i\frac{\partial}{\partial x_i}$. The orientation of $S^n$ is given by $\omega=\iota^*(\iota_X(\Omega))$, where $\iota$ is the inclusion map. What we want to show is that $\alpha^*\omega=\omega$. First, note that $A\circ \iota=\iota\circ\alpha$, so 
    \begin{align*}
        \alpha^*\omega=&\alpha^*\iota^*(\iota_X\Omega)
        =\iota^*A^*(\iota_X\Omega).
    \end{align*}
    Also, $A_*X_{p}=X_{A(p)}$ by direct computation, and $A^*\Omega=(-1)^{n+1}\Omega$, using the distributive property of the pullback and that it commutes with $d$. Therefore, \begin{align*}
        A^*(\iota_X\Omega)_p(v_1, \dots, v_n)=&(\iota_X\Omega)_{A(p)}(A_*v_1, \dots, A_*v_n)\\
        =&\Omega_{A(p)}(X_{A(p)}, A_*v_1, \dots, A_*v_n)\\
        =&\Omega_{A(p)}(A_*X_p, A_*v_1, \dots, A_*v_n)\\
        =&(A^*\Omega)_p(X_p, v_1, \dots, v_n)\\
        =&(-1)^{n+1}\Omega_p(X_p, v_1, \dots, v_n)\\
        =&(-1)^{n+1}(\iota_X\Omega)_p(v_1, \dots, v_n).
    \end{align*}
    Thus, \[\alpha^*\omega=\iota^*A^*(\iota_X\Omega)=(-1)^{n+1}\iota^*(\iota_X\Omega)=(-1)^{n+1}\omega,\]
    so $\alpha$ is orientation preserving if and only if $n$ is odd.
    
    $5\implies 1$: If $n$ is odd, we obtain a continuous map $V:p\mapsto ip$ by considering points in $S^n$ as points in $\mathbb{C}^{(n+1)/2}$. Then, we get a non-vanishing vector field $X_p=\iota^{-1}(V(p))=\gamma_*(\frac{d}{dt}|_{t=0})$, where $\gamma(t)=e^{it}p$.

\end{proof}

\begin{thebibliography}{9}
    \bibitem{tu}
    Loring W. Tu (2010) \emph{An Introduction to Manifolds}, Springer.
    
    \bibitem{lee}
    John M. Lee (2013) \emph{Introduction to Smooth Manifolds}, Springer.
\end{thebibliography}

\end{document}