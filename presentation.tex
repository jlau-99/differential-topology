\documentclass[9pt]{beamer}
%\usepackage{amsmath,amssymb,amsthm,tikz-cd}
\usetheme{Madrid}
\usecolortheme{default}
\DeclareMathOperator{\Hom}{Hom}

\title[Differentiable manifolds and the Hairy Ball Theorem]
{Differentiable manifolds and the Hairy Ball Theorem}

\author[Jonathan Lau] % (optional)
{Jonathan Lau}

\AtBeginSection[]
{
	\begin{frame}
		\frametitle{Table of Contents}
		\tableofcontents[currentsection]
	\end{frame}
}

\begin{document}
	
\frame{\titlepage}

\begin{frame}
	\frametitle{Table of Contents}
	\tableofcontents
\end{frame}

\section{Manifolds}
	
\begin{frame}
    \begin{block}{Manifolds}
        A $n$ dimensional manifold is a subset of $\mathbb{R}^\ell$ such that for each point $p$, there is a neighborhood $U$ and a homeomorphism $\phi:U\rightarrow V\subset \mathbb{R}^n$ such that $\phi$ and $\phi^{-1}$ are smooth.
    \end{block}
\end{frame}

\begin{frame}
    \begin{block}{Tangent Space}
        Given a point $p\in M$ and a smooth curve $\gamma:(-1,1)\rightarrow M$ in $M$ such that $\gamma(0)=p$, its velocity vector is $\frac{d\gamma}{dt}\vert_{t=0}$. The set of all velocity vectors at $p$ is the tangent space at $p$, denoted $T_p M$.
    \end{block}
    If we fix a neighborhood $U$ of $p$ and a $\phi:U\rightarrow V\subset \mathbb{R}^n$ centered at $p$, construct curves $\gamma_i:t\mapsto \phi^{-1}\circ\iota_i(t)$, where $\iota_i$ is the inclusion into the $i$th coordinate. Their velocity vectors form a basis for the tangent space.

    Example: sphere
\end{frame}

\begin{frame}
    \begin{block}{1-form}
        A 1-form $\omega$ is a linear function from $T_p M$ to $\mathbb{R}$. $\hom(T_p M, \mathbb{R})$ is a vector space of dimension $n$, with basis $dx^i$, $dx^i(a_1, \dots, a_n) = a_i$.
    \end{block}

    \begin{block}{$k$-form}
        A $k$-form $\omega$ is an alternating multilinear function from $(T_p M)^k$ to $\mathbb{R}$.
    \end{block}
\end{frame}

\begin{frame}
    \begin{block}{Wedge Product}
        The wedge product of $k$ 1-forms $\omega_1, \dots, \omega_k$ is the k-form $\omega^1\wedge\dots\wedge\omega^k(v_1, \dots, v_k)=det(\omega_i(v_j))$.
    \end{block}

    \begin{block}{Theorem}
        $\{dx^{i_1}\wedge\dots\wedge dx^{i_k}|1\leq i_i<\dots<i_k\leq n\}$ is a basis for $\Lambda^k(T_pM)$, the set of alternating multilinear functions on $(T_pM)^k$.
    \end{block}

    Example: On $\mathbb{R}^3$, the form $dx\wedge dy$ applied to $(1,2,3),(0,1,2)$ is $det((1,2)^T(0,1)^T)=1$. We see that 
\end{frame}

\begin{frame}
    \begin{block}{stuff}
        stuff ...
    \end{block}
\end{frame}

\section{Stokes' Theorem}

\begin{frame}
    \begin{block}{stuff}
        stuff ...
    \end{block}
\end{frame}

\section{The Hairy Ball Theorem}

\begin{frame}
    \begin{block}{stuff}
        stuff ...
    \end{block}
\end{frame}

\end{document}